\newpage
\section*{Abstract}
Underwater propulsion systems include screw propeller propulsion and tail fin propulsion using fish-like robots. Screw propellers provide high propulsion efficiency 
both in the water and underwater, and they are widely used in ships and other vessels. However, from an ecosystem research perspective, screw propellers can affect 
the surrounding environment and are not ideal for surveys. On the other hand, tail fin propulsion has less environmental impact and offers excellent acceleration 
and maneuverability, enabling rapid movement toward a target while avoiding obstacles. These advantages make it well-suited for disaster relief, such as in flooding, 
as well as for studying the ecosystems of underwater organisms.
As a result, the development of fish-shaped robots has gained attention for both disaster support and underwater ecosystem research. In our laboratory, we have developed 
various fish-shaped robots. Specifically, last year, we successfully developed a wire-driven fish robot that enables supple, fish-like movements, as well as a waterproof 
fish robot with a streamlined, fish-like body and flexible outer skin. However, the wire-driven system had gaps between the links, meaning water could not be splashed off
 the body. In the flexible skin model, the skeletal links and skin were not synchronized, resulting in limited movement of the body.
In this study, we combined these two approaches to develop a wire-driven fish-shaped robot with a flexible outer skin. This innovation allows for supple, fish-like movement 
while also enabling the robot to splash water off its torso, thus improving swimming performance. The swimming experiments confirmed that the wire-driven system achieved
 fish-like movements, with the flexible outer skin moving in accordance with the link movements. Additionally, we conducted straight-line swimming experiments with and 
 without the outer skin to assess its impact on swimming performance. The results showed that the swimming speed increased when the outer skin was attached. This increase 
 in speed is likely due to the robot’s body becoming more streamlined when the skin was added. Therefore, having a streamlined body is highly effective in improving swimming 
 speed.