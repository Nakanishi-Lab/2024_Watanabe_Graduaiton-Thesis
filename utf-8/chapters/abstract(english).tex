\newpage
\section*{Abstract}
The propulsion system in water includes methods such as using screw propellers for propulsion and robot fish that mimic fish's tail fin 
propulsion. Screw propellers are widely used in ships and other watercraft due to their high propulsion performance both on and under 
the water. However, from the perspective of ecological surveys, screw propellers are not ideal as they can impact the surrounding 
environment, making them unsuitable for research. On the other hand, tail fin propulsion has minimal impact on the environment and offers 
excellent acceleration and maneuverability, enabling swift movement while avoiding obstacles to reach a desired location. Therefore, 
the development of fish-like robots has garnered attention for disaster support, underwater biological ecosystem surveys, and similar 
applications.
In our laboratory, we have developed various types of fish-like robots. Specifically, last year, we succeeded in developing a wire-driven 
fish robot that enables smooth, fish-like movement, and a flexible outer-skin fish robot that has a streamlined body and is completely 
waterproof. However, the wire-driven type had gaps between the links, and the body could not spray water, while the flexible outer-skin 
type faced the issue of the skeleton links and outer skin not being synchronized, causing the body to be unable to swim.
In this study, we combined these two designs to develop a wire-driven fish-like robot with a flexible outer skin. This approach 
allows for fish-like, flexible movement while ensuring that water can be sprayed from the body, improving swimming performance. 
In swimming experiments, we confirmed that the wire-driven mechanism achieved fish-like, fluid motion and that the flexible outer 
skin followed the movement of the links. We also conducted straight-line swimming tests both with and without the flexible outer 
skin to examine its impact on swimming performance. The results showed that the swimming speed improved when the flexible outer 
skin was used. The reason for this speed increase is likely because the body became more streamlined with the addition of the 
flexible outer skin. Therefore, equipping the robot with a streamlined body is highly effective in improving swimming speed.