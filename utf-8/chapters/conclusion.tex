\newpage
\section{結言}
本研究では柔軟外皮を有するワイヤ駆動式魚型ロボットを開発し,外皮による遊泳性能の検証を行った.遊泳実験ではリンクの動きに追従して
外皮が動作することを確認でき,先行研究\cite{kyu}での課題を解決できた.また,シリコンによって完全な防水を実現でき,これによって
頭部のメンテナンス性も向上し,バッテリー交換も容易に行うことができた.

実験結果からは,外皮を装着することにより遊泳速度が向上し,尚且つ高周波域において遊泳性能に大きな影響を与えることがわかった.
このことから,外皮を装着して流線形のボディを魚ロボットに備えさせることは,遊泳性能を向上させるのに非常に有効であると考えられる.

今後は急旋回実験や自由軌道実験を行い,さらなる性能の検証や泳ぎ方による遊泳性能への影響を検証する必要性がある.
また,アクチュエータを変更することでさらに推進性能を向上させ,高速遊泳の実現を目指す.
そして最終的にはウロコを装着して,魚らしい見た目を持った高機動性を有する魚ロボットの開発を目指す.
