\newpage
\setcounter{page}{1}
\section{緒言}
水中の推進システムにはスクリュープロペラを用いた推進方法や魚を模したロボットによる尾びれ推進などがあげられる\cite{ichi}.スクリュープロペラは水上,水中における推進性能が高く,
船舶などに広く用いられている.しかし,生態系調査の面で考えると,スクリュープロペラは周辺の植物や水中動物などを巻き込んで水中の環境に影響を与え,騒音によって周辺生物を
驚かせるなど,調査に適しているとは言えない.その一方で尾びれ推進は周辺生物を巻き込むなど環境に影響を与えにくく,尚且つ加速性・旋回性に優れているため障害物を避けながら
目的の地点まで高機動な遊泳を可能にする.以上のことから水害などの災害支援,水中生物の生態系調査の面で魚型ロボットの開発は注目されている\cite{ni}\cite{san}.

そこで我々の研究室ではこれまで様々な魚型ロボットを開発してきた.駆動機構として弾性体を用いた飛び移り座屈機構を採用した魚ロボット\cite{yon}\cite{go}や,それに加えて屈曲可能な胴体
構造を有する魚ロボット\cite{roku,nana,hachi}の開発に成功してきた.さらに昨年度卒業研究では駆動機構としてワイヤ駆動方式を用いたより魚らしい形状を持つ魚ロボットの開発に成功し,
魚のように胴体を屈曲させてしなやかに遊泳を行うことができた.しかし,これらの魚ロボットには弾性体に追従する骨格リンクの隙間に水が入り込み,結果的に弾性体のみで水をかいてしまい,体全体でかく
ことができていないという課題があった.

一方で同じく昨年度卒業研究(先行研究\cite{kyu})では胴体部全体をシリコン製の柔軟外皮で覆った魚型ロボットの開発が行われた.柔軟外皮は魚らしい流線型のフォルムの実現のみならず,従来のOリングを
用いた手法よりも容易かつ確実性の高い防水性能を実現した.しかし,遊泳に関しては骨格リンクの動きに柔軟外皮が追従せず,胴体部を振って泳ぐことはできなかった.そのため遊泳速度が大きく低下し,柔軟
外皮が遊泳速度に対してどのように影響を与えるのか検証することができなかった.

そこで本研究では,しなやかな遊泳が可能なワイヤ駆動方式と,流線形の胴体フォルムを実現可能な柔軟外皮を組み合わせた魚型ロボットを開発する.ロボットはアジのスキャンデータを基に作成し,先行研究
\cite{juu}によって提唱されたアジ型遊泳の定義のもとに,胴体後ろ半分を屈曲させることで遊泳を行う.そして柔軟外皮あり・なしで直進遊泳実験を行い,柔軟外皮が遊泳性能に与える影響について検証・考察を行う.
