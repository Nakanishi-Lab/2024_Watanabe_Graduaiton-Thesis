\newpage
\setcounter{page}{1}
\section{緒言}
水中の推進システムにはスクリュープロペラを用いた推進方法や魚を模したロボットによる尾びれ推進などがあげられる\cite{ichi}.スクリュープロペラは水上,水中における推進性能が高く,
船舶などに広く用いられている.しかし,生態系調査の面で考えると,スクリュープロペラは周辺の植物や水中動物などを巻き込んで水中の環境に影響を与え,騒音によって周辺生物を
驚かせるなど,調査に適しているとは言えない.その一方で尾びれ推進は周辺生物を巻き込むなど環境に影響を与えることがなく,尚且つ加速性・旋回性に優れているため障害物を避けながら
目的の地点まで高機動な遊泳を可能にする.以上のことから水害などの災害支援,水中生物の生態系調査の面で魚型ロボットの開発は注目されている\cite{ni}\cite{san}.

そこで我々の研究室ではこれまで様々な魚型ロボットを開発してきた.駆動機構として弾性体を用いた飛び移り座屈機構を採用した魚ロボット\cite{yon}\cite{go}や,それに加えて屈曲可能な胴体
構造を有する魚ロボット\cite{roku}\cite{nana}\cite{hachi}の開発に成功してきた.さらに昨年度は駆動機構としてワイヤ駆動方式を用いたより魚らしい形状を持つ魚ロボットの開発に成功し,魚のように胴体を屈曲させてしなやかに
遊泳を行うことができた.しかし,これらの魚ロボットには弾性体に追従する骨格リンクの隙間に水が入り込み,結果的に弾性体のみで水をかいてしまい,体全体でかくことができていないという課題があった.

それに対して昨年度先行研究\cite{kyu}ではリンク間にできる隙間に水が入り込まないように柔軟外皮を装着した魚ロボットの開発を行った.シリコン製の柔軟外皮はリンク間に水を入り込ませること無く,
水中において絶大な防水機能を発揮し,長年の問題であった防水をOリングを用いていた時よりも比較的容易に行うことができるようになった.しかし,遊泳に関しては骨格リンクの動きに柔軟外
皮が追従せず,リンクの動作を外皮にうまく伝えきれなかった.その影響で従来の魚ロボットよりも遊泳速度が低下してしまい,外皮が遊泳速度に対してどのように影響を与えるのか検証することが
できなかった.

そこで本研究では,魚らしい形状を持ち,しなやかな遊泳を可能にするワイヤ駆動方式の魚型ロボットに,骨格リンクの動作に追従できる柔軟外皮を装着した魚型ロボットを開発する.本機体は
アジのスキャンデータを基に作成し,先行研究によって提唱されたアジ型遊泳の定義のもとに,胴体後ろ半分を屈曲させることで遊泳を行う.そして開発した機体を用いて外皮あり・なしで直進
遊泳実験を行い,外皮が遊泳性能に与える影響について検証・考察を行う.そして今後はアクチュエータをサーボモータからさらに細かな制御が可能なブラシレスモータに変更し,魚が持つウロコを
柔軟外皮装着させることで魚らしい見た目と高機動性を有する魚型ロボットの開発を目指す.
