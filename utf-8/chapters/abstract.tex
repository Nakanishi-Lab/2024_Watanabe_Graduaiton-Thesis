\newpage
\section*{概要}
水中の推進システムにはスクリュープロペラを用いた推進方法や魚を模したロボットによる尾びれ推進などがあげられる.スクリュープロペラは水上,水中における推進性能が高く,
船舶などに広く用いられている.しかし,生態系調査の面で考えると,スクリュープロペラは周辺環境に影響を与え,調査に適しているとは言えない.その一方で尾びれ推進は周辺環境に影響を与えることがなく,尚且つ加速性・旋回性に優れているため障害物を避けながら目的の地点まで速やかな移動を可能にする.以上のことから水害などの災害支援,水中生物の
生態系調査の面で魚型ロボットの開発は注目されている.そこで我々の研究室ではこれまで様々な魚型ロボットを開発してきた.特に昨年度は魚らしくしなやかな動きを可能にするワイヤ駆動
式の魚ロボットと,完全防水可能かつ魚らしい流線形のボディを有する柔軟外皮装着型の魚ロボットの開発に成功した.しかし,昨年度の先行研究には次のような課題があった.ワイヤ駆動式の
魚ロボットには骨格リンク間にできる隙間に水が入り込み,体全体を使って水をかけないといった課題が,柔軟外皮装着型のロボットには骨格リンクの動きに外皮が追従せず,十分な遊泳性能を
得られなかったという課題があった.そこで本研究ではワイヤ駆動式の魚ロボットに柔軟外皮を装着することによって,魚らしいしなやかな動きを可能にし,かつ骨格リンクの動きに外皮が追従
できるようにすることで十分な遊泳性能を発揮できる魚ロボットを開発した.また,外皮あり・なしそれぞれで直進遊泳実験を行い,外皮による遊泳性能に与える影響について検証を行った.遊
泳実験の結果,外皮を装着することによって遊泳速度が向上することがわかった.速度が向上した要因としては,ロボットのボディが外皮を装着することによって流線形になったことが考えられ
る.したがって流線形のボディを備えることは,遊泳速度を向上させるのに非常に有効であると考えられる.
\newpage
\section*{Abstract}
In underwater propulsion systems, methods such as using screw propellers and fish-like robots with tail fin propulsion are commonly mentioned. Screw propellers 
have high propulsion performance both on the water surface and underwater, and are widely used in ships and other vessels. However, from the perspective of ecosystem 
surveys, screw propellers are not ideal, as they can impact the surrounding environment and are not suitable for surveys. On the other hand, tail fin propulsion 
does not affect the surrounding environment, and its excellent acceleration and maneuverability enable swift movement to a target location while avoiding obstacles. 
For these reasons, the development of fish-like robots has attracted attention for applications in disaster response, such as in the case of flooding, and for aquatic 
ecosystem surveys.
In our laboratory, we have developed various types of fish-like robots. In particular, last year we succeeded in developing a wire-driven fish robot that enables 
flexible movements similar to a real fish, as well as a soft-skin fish robot with a streamlined body and complete waterproofing. However, there were some challenges 
in last year's research. The wire-driven fish robot faced the issue of water entering the gaps between the skeletal links, which prevented efficient water management 
across the entire body. The soft-skin robot encountered the problem that the skin did not follow the movements of the skeletal links, leading to insufficient swimming 
performance.
Therefore, in this study, we developed a fish robot by attaching a soft skin to the wire-driven robot, enabling flexible, fish-like movements while allowing the skin 
to follow the movement of the skeletal links. This development ensures sufficient swimming performance. Additionally, we conducted swimming experiments with and without 
the soft skin to investigate the impact of the soft skin on swimming performance. The results of these experiments showed that the swimming speed increased when the soft 
skin was attached. The increase in speed is believed to be due to the streamlined body shape achieved by the soft skin. Hence, it can be concluded that having a streamlined 
body is highly effective in improving swimming speed.
