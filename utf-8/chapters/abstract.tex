\newpage
\section*{概要}
水中の推進システムにはスクリュープロペラを用いた推進方法や魚を模したロボットによる尾びれ推進などがあげられる.スクリュープロペラは水上,水中における推進性能が高く,
船舶などに広く用いられている.しかし,生態系調査の面で考えると,スクリュープロペラは周辺環境に影響を与え,調査に適しているとは言えない.その一方で尾びれ推進は周辺環境に影響を
与えにくく,尚且つ加速性・旋回性に優れているため障害物を避けながら目的の地点まで速やかな移動を可能にする.以上のことから水害などの災害支援,水中生物の
生態系調査の面で魚型ロボットの開発は注目されている.そこで我々の研究室ではこれまで様々な魚型ロボットを開発してきた.特に昨年度は魚らしくしなやかな動きを可能にするワイヤ駆動
式の魚ロボットと,完全防水可能かつ魚らしい流線形のボディを有する柔軟外皮装着型の魚ロボットの開発に成功した..しかし,ワイヤ駆動式ではリンク間に隙間があり胴体部で水をかけていな
いこと,柔軟外皮装着型では骨格リンクと外皮が連動せず胴体部が動かないという問題があった.そこで本研究ではこれら2つの研究を組み合わせ,柔軟外皮を備えたワイヤ駆動式魚型ロボットを
開発した.これにより魚らしいしなやかな動きを可能にしつつ,胴体部でも水をかけるようにすることで遊泳性能の向上を目指した.遊泳実験においては,ワイヤ駆動によって魚らしくしなやかな
動作を実現し,かつ柔軟外皮がリンクの動きに追従して動作することを確認できた.また外皮あり・なしそれぞれで直進遊泳実験を行い,外皮による遊泳性能に与える影響について検証を行った.
結果として,外皮を装着することによって遊泳速度が向上することを確認した.速度が向上した要因としては,ロボットのボディが外皮を装着することによって流線形になったことが考えられる.
したがって流線形のボディを備えることは,遊泳速度を向上させるのに非常に有効であると考えられる.

\newpage
\section*{Abstract}
Underwater propulsion systems include screw propeller propulsion and tail fin propulsion using fish-like robots. Screw propellers provide high propulsion efficiency 
both in the water and underwater, and they are widely used in ships and other vessels. However, from an ecosystem research perspective, screw propellers can affect 
the surrounding environment and are not ideal for surveys. On the other hand, tail fin propulsion has less environmental impact and offers excellent acceleration 
and maneuverability, enabling rapid movement toward a target while avoiding obstacles. These advantages make it well-suited for disaster relief, such as in flooding, 
as well as for studying the ecosystems of underwater organisms.
As a result, the development of fish-shaped robots has gained attention for both disaster support and underwater ecosystem research. In our laboratory, we have developed 
various fish-shaped robots. Specifically, last year, we successfully developed a wire-driven fish robot that enables supple, fish-like movements, as well as a waterproof 
fish robot with a streamlined, fish-like body and flexible outer skin. However, the wire-driven system had gaps between the links, meaning water could not be splashed off
 the body. In the flexible skin model, the skeletal links and skin were not synchronized, resulting in limited movement of the body.
In this study, we combined these two approaches to develop a wire-driven fish-shaped robot with a flexible outer skin. This innovation allows for supple, fish-like movement 
while also enabling the robot to splash water off its torso, thus improving swimming performance. The swimming experiments confirmed that the wire-driven system achieved
 fish-like movements, with the flexible outer skin moving in accordance with the link movements. Additionally, we conducted straight-line swimming experiments with and 
 without the outer skin to assess its impact on swimming performance. The results showed that the swimming speed increased when the outer skin was attached. This increase 
 in speed is likely due to the robot’s body becoming more streamlined when the skin was added. Therefore, having a streamlined body is highly effective in improving swimming 
 speed.