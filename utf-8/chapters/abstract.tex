\newpage
\section*{概要}
水中の推進システムにはスクリュープロペラを用いた推進方法や魚を模したロボットによる尾びれ推進などがあげられる.スクリュープロペラは水上,水中における推進性能が高く,
船舶などに広く用いられている.しかし,生態系調査の面で考えると,スクリュープロペラは周辺環境に影響を与え,調査に適しているとは言えない.その一方で尾びれ推進は周辺環境に影響を
与えにくく,尚且つ加速性・旋回性に優れているため障害物を避けながら目的の地点まで速やかな移動を可能にする.以上のことから水害などの災害支援,水中生物の
生態系調査の面で魚型ロボットの開発は注目されている.そこで我々の研究室ではこれまで様々な魚型ロボットを開発してきた.特に昨年度は魚らしくしなやかな動きを可能にするワイヤ駆動
式の魚ロボットと,完全防水可能かつ魚らしい流線形のボディを有する柔軟外皮装着型の魚ロボットの開発に成功した.しかし,ワイヤ駆動式ではリンク間に隙間があり胴体部で水をかけていな
いこと,柔軟外皮装着型では骨格リンクと外皮が連動せず胴体部が動かないという問題があった.そこで本研究ではこれら2つの研究を組み合わせ,柔軟外皮を備えたワイヤ駆動式魚型ロボットを
開発した.これにより魚らしいしなやかな動きを可能にしつつ,胴体部でも水をかけるようにすることで遊泳性能の向上を目指した.遊泳実験においては,ワイヤ駆動によって魚らしくしなやかな
動作を実現し,かつ柔軟外皮がリンクの動きに追従して動作することを確認できた.また柔軟外皮あり・なしそれぞれで直進遊泳実験を行い,柔軟外皮が遊泳性能に与える影響について検証を行った.
結果として,柔軟外皮を装着することによって遊泳速度が向上することを確認した.速度が向上した要因としては,ロボットのボディが柔軟外皮を装着することによって流線形になったことが考えられる.
したがって流線形のボディを備えることは,遊泳速度を向上させるのに非常に有効であると考えられる.